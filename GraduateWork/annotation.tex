Олимпиады по программированию сегодня "--- очень распространённое и актуальное явление. Они проводятся как среди школьников, так и среди студентов и профессиональных работников, как на уровне вузов, так и на международном уровне. Многие крупные IT-компании, такие как Google, Facebook, Яндекс, Mail.ru, ВКонтакте, регулярно проводят свои собственные олимпиады.

Проведение олимпиад по программированию требует наличия специального программного обеспечения: тестирующей системы, удовлетворяющей определённому набору требований, а также "--- системы для разработки задач. Подготовка наборов тестов для задач "--- весьма трудоёмкий процесс, требующий применения специальных инструментов, таких как генераторы, валидаторы, чекеры и интеракторы.

В данной работе мы поставим перед собой цель изучить основные аспекты работы тестирующих систем и систем разработки задач для олимпиад по программированию, а также "--- написать программу с удобным графическим пользовательским интерфейсом, предоставляющую возможность разрабатывать новые задачи и проверять решения на подготовленных тестах. Предполагается написание программы на языке Java с использованием библиотеки Swing.

%Алгоритмы неточного сопоставления строк среди всех строковых алгоритмов представляют собой целую группу важнейших и часто используемых нами в повседневности методов для решения возникающих перед нами задач. Наряду с точным сравненим строк их неточное сопоставление находит не менее важное применение во многих областях.

%Суть алгоритмов данной группы "--- поиск оптимального выравнивания строковых последовательностей, соответствующего тому или иному заданному критерию. В число задач, решаемых с помощью данных алгоритмов, входят поиск редакционного расстояния между строками, их сходства, их наибольшей общей подпоследовательности, локального выравнивания строк, а также поиск оптимального глобального выравнивания с учётом пропусков.

%Целью данной работы является реализация основных алгоритмов неточного сопоставления строк, решающих задачи, перечисленные выше. Предполагается создание приложения с графическим пользовательским интерфейсом на основе библиотеки Swing.