На олимпиадном соревновании по программированию участникам предлагается решить несколько задач, каждая из которых представлена набором тестов "--- входных и выходных данных алгоритма, который нужно реализовать. Тесты подготавливаются заранее, и их подготовка "--- довольно длительный процесс. Он подразумевает генерацию входных данных для тестов, их валидацию, а также генерацию выходных данных, написание чекеров и иногда "--- интеракторов \cite{testlib}.

Полное решение задачи "--- работающая программа, которая на каждом тесте даёт правильный ответ. За время соревнования каждому участнику разрешается делать неограниченное количество посылок по каждой задаче различных версий своего решения в тестирующую систему, которая автоматически компилирует решения, запускает их на тестах и выносит для каждой посылки вердикт по определённым правилам. Таким образом, одна посылка участника "--- это файл с исходным кодом программы, вердикт, который тестирующая система присвоила этой посылке, а также "--- общая информация о посылке (максимальное время работы программе на одном тесте, количество использованной памяти и т. д.).

В зависимости от правил проведения того или иного соревнования, участникам могут быть видны вердикты по посылкам, которые они отправляют в тестирующую систему, либо не видны. Часто окончательным решением участника той или иной задачи считается та посылка, которую он отослал позже остальных. Кроме того, участникам может быть доступен общий монитор соревнования, в котором видны текущие результаты по каждой задаче всех остальных участников.

По определённым правилам, в зависимости от используемой на соревновании системы оценивания (ICPC, IOI \cite{wiki}), каждой задаче присваивается некоторое количество баллов. Они суммируются для каждого участника, и на основе них затем формируется итоговая таблица результатов соревнования.

Таким образом, разработка задач и проверка решений "--- два взаимосвязанных процесса, но рассматривать их можно отдельно друг от друга. Исследуем каждый из них более подробно.