\subsection*{Принципы разработки}

Разработка задач "--- это процесс, который происходит задолго до самого соревнования, и его цель "--- подготовить корректные тесты, учитывающие все частные случаи задач. Тестов может быть много, и входные данные каждого из них могут иметь большой размер, поэтому для их генерации пишутся специальные программы "--- генераторы. Для того, чтобы проверить, что входные данные того или иного теста корректны для данной задачи, пишутся валидаторы. Чекеры используются для того, что проверять, являются ли выходные данные верными, в тех случаях, когда у задачи может быть много правильных ответов. И, наконец, интеракторы используются в интерактивных задачах, где программа-решение участника может общаться с тестирующей системой, следуя определённому формату ввода-вывода. Последние, впрочем, встречаются нечасто, поэтому мы не будем рассматривать их в нашей работе.

Каждое описанное средство "--- в свою очередь тоже программа. Очевидно, что в каждой из них для многих олимпиадных задач приходится выполнять множество похожих операций. Это подразумевает, что в данном случае целесообразно использование специальной библиотеки, предоставляющей удобные средства для решения типичных задач. Одной из таких библиотек является Testlib \cite{testlib}, написанная на языке C++. Именно на её характеристики мы будем опираться при рассмотрении средств разработки задач и в дальнейшем при их написании.

\subsection*{Тесты}

Тесты

\subsection*{Генераторы}

Генераторы

\subsection*{Валидаторы}

Валидаторы

\subsection*{Чекеры}

Чекеры