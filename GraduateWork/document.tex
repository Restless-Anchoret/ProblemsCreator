\begin{document}

\thispagestyle{empty}

{\large
\begin{center}
МИНОБРНАУКИ РОССИИ\par
ФЕДЕРАЛЬНОЕ ГОСУДАРСТВЕННОЕ БЮДЖЕТНОЕ\par
ОБРАЗОВАТЕЛЬНОЕ УЧРЕЖДЕНИЕ ВЫСШЕГО ПРОФЕССИОНАЛЬНОГО ОБРАЗОВАНИЯ\par
<<ВОРОНЕЖСКИЙ ГОСУДАРСТВЕННЫЙ УНИВЕРСИТЕТ>>\par
(ФГБОУ ВПО <<ВГУ>>)\par
\vspace{10mm}

Факультет прикладной математики, информатики и механики\par
\vspace{5mm}
Кафедра математического обеспечения ЭВМ\par
\vspace{22mm}

\textbf{Разработка задач для олимпиад по программированию}\par
\vspace{10mm}

Дипломная работа\par
\vspace{5mm}
по направлению 010500 <<Прикладная математика и информатика>>\par
\vspace{13mm}
\end{center}

Студент 4 курса 8 группы \underline{\qquad\qquad\qquad} Кузнецов~И.~Э.\par
\qquad\qquad\qquad\qquad\qquad\qquad\quad(подпись)\par
\vspace{2mm}
Зав. кафедрой \underline{\qquad\qquad\qquad} д.~ф.-м.~н.,~доц.~Махортов~С.~Д.\par
\qquad\qquad\qquad\qquad(подпись)\par
\vspace{2mm}
Руководитель \underline{\qquad\qquad\qquad} д.~ф.-м.~н.,~доц.~Махортов~С.~Д.\par
\qquad\qquad\qquad\qquad(подпись)\par
\vspace{23mm}

\begin{center}
Воронеж 2016
\end{center}}

\chapter*{Аннотация}
Текст аннотации.

%Алгоритмы неточного сопоставления строк среди всех строковых алгоритмов представляют собой целую группу важнейших и часто используемых нами в повседневности методов для решения возникающих перед нами задач. Наряду с точным сравненим строк их неточное сопоставление находит не менее важное применение во многих областях.

%Суть алгоритмов данной группы "--- поиск оптимального выравнивания строковых последовательностей, соответствующего тому или иному заданному критерию. В число задач, решаемых с помощью данных алгоритмов, входят поиск редакционного расстояния между строками, их сходства, их наибольшей общей подпоследовательности, локального выравнивания строк, а также поиск оптимального глобального выравнивания с учётом пропусков.

%Целью данной работы является реализация основных алгоритмов неточного сопоставления строк, решающих задачи, перечисленные выше. Предполагается создание приложения с графическим пользовательским интерфейсом на основе библиотеки Swing.

\renewcommand{\contentsname}{Содержание}
\tableofcontents

\chapter*{Введение}
\addcontentsline{toc}{chapter}{Введение}
Текст введения.

%Алгоритмы неточного сопоставления строк представляют собой целую группу важнейших и часто используемых нами в повседневности методов для решения возникающих перед нами задач. Кратко перечислим известные применения алгоритмов неточного сопоставления, чтобы подчеркнуть их нынешнюю актуальность.

%Наиболее часто упоминаемое применение данных алгоритмов "--- в области биоинформатики. Дэн Гасфилд в своей книге \cite{gasfild} подчёркивает важность некоего <<первого факта анализа биологических последовательностей>>, позволяющего применять строковые алгоритмы к биологическим последовательностям и получать в результате выводы, имеющие большую научную ценность. Вот как Дэн Гасфилд формулирует данный факт: <<В биомолекулярных последовательностях (ДНК, РНК или аминокислотных последовательностях) высокое сходство последовательностей обычно влечёт существенное функциональное или структурное сходство>>.

%Биомолекулярные последовательности можно представлять в виде длинных цепочек из азотистых оснований (которых может быть всего четыре -- аденин, гуанин, цитозин и тимин). Таким образом, к ним можно применять все строковые алгоритмы. Однако заметим, что применение к этим последовательностям алгоритмов неточного сопоставления позволяет решить некоторые проблемы, которые точное сравнение решить бы не смогло, а именно "--- <<наличие ошибок в молекулярных данных>> и <<активные мутационные процессы>> ДНК \cite{gasfild}.

%Сегодня существуют целые базы данных ДНК, содержащие в себе огромные объёмы информации о биомолекулярных последовательностях. После расшифровки нового гена принято переводить его в последовательность аминокислот и проверять на сходство с записями баз данных. Как пишет Дэн Гасфилд \cite{gasfild}, <<Сейчас никому в голову не придёт публиковать расшифровку вновь клонированного гена без проведения такого поиска>>. Таким образом, неточное сопоставление строковых последовательностей занимает в биоинформатике центральное место.

%Но, конечно, кроме этого насчитывается также ещё множество различных применений данной группы алгоритмов. Они активно применяются для исправления ошибок в словах в поисковых системах, базах данных, при вводе текста и при автоматическом распознавании отсканированного текста или речи. Подобное исправление ошибок стало привычным в современном мире, где почти у каждого человека есть свой смартфон, которым пользуются изо дня в день. Кроме того, алгоритмы неточного сравнения строк используются также для сравнения текстовых файлов, где в качестве символов выступают строки, а в качестве строк "--- файлы \cite{wiki_dist}.

%Мы будем рассматривать данные алгоритмы без жёсткой привязки к какому-либо их конкретному применению. Наше исследование будет носить сугубо теоретический характер и будет относиться к сравнению строк вообще, каких бы то ни было. Нами будет рассмотрено и реализовано несколько известных алгоритмов неточного сравнения пар строк, будет приведено их краткое теоретическое описание и оценена их временная сложность.

\chapter{Общетеоретическая часть}
\section{Основные понятия}
Основные понятия.
\section{Правила проведения олимпиад}
Правила проведения олимпиад.
\section{Инструменты разработки задач}
Интрументы разработки задач.
\section{Запуск решений}
Запуск решений.

\chapter{Практическая часть}
\section{Постановка задачи}
Постановка задачи.
\section{Средства реализации}
Средства реализации.
\section{Тестирующий модуль}
Тестирующий модуль.
\section{Библиотека для разработки задач}
Библиотека для разработки задач.
\section{Модуль для работы с файловой системой}
Модуль для работы с файловой системой.
\section{Модуль с графическим интерфейсом}
Модуль с графическим интерфейсом.
\section{Пример работы приложения}
Пример работы приложения.

\chapter*{Заключение}
\addcontentsline{toc}{chapter}{Заключение}
Текст заключения.

%В ходе курсовой работы было исследовано четыре различных алгоритма поиска выравниваний пар строк, обладающих определёнными характеристиками. Мы подробно обсудили, как используется в них принцип динамического программирования и каким образом строится искомое оптимальное выравнивание, поговорили об эффективности каждого из этих алгоритмов, а также дополнительно для двух из них выявили способ нахождения количества кооптимальных выравниваний.

%Была поставлена задача написать приложение, реализующее все четыре алгоритма и позволяющее запускать их с некоторым набором входных параметров. В ходе работы это приложение было разработано, в процессе чего активно использовались объектно-ориентированное программирование и паттерны проектирования. Была подробно описана структура классов, использованных в приложении, и приведён пример его работы.

%Ясно, что в созданном приложении реализованы лишь некоторые алгоритмы, работающие с выравниваниями строковых последовательностей, и что помимо них существует ещё множество подобных алгоритмов, из чего следует возможность дальнейшей разработки данного приложения.

\renewcommand{\bibname}{Список использованных источников}
\begin{thebibliography}{9}
\addcontentsline{toc}{chapter}{Список использованных источников}
%\bibitem{gasfild} Гасфилд Дэн. Строки, деревья и последовательности в алгоритмах: Информатика и вычислительная биология / Пер. с англ. И.~В.~Романовского.~"--- СПб.~: Невский Диалект; БХВ-Петербург, 2003.~"--- 654~с.
%\bibitem{smith} Смит, Билл. Методы и алгоритмы вычислений на строках.~: Пер. с англ.~"--- М.~: ООО <<И.~Д.~Вильямс>>, 2006.~"--- 496~с.
%old one
\bibitem{wiki_olymp} Wikipedia, статьи.\par
(URL: https://ru.wikipedia.org/wiki/Олимпиады\_по\_программированию)\par
%(URL: https://ru.wikipedia.org/wiki/Международная\_студенческая\_олимпиада\_по\_программированию)\par
%(URL: https://ru.wikipedia.org/wiki/Выравнивание\_последовательностей)\par
%(URL: https://ru.wikipedia.org/wiki/Расстояние\_Левенштейна)
\bibitem{gamma} Приёмы объектно-ориентированного проектирования. Паттерны проектирования. Гамма Э. [и др.].~"--- СПб~: Питер, 2001.~"--- 368~с.
\bibitem{java} Java, документация.\par
(URL: http://docs.oracle.com/javase/8/docs/)
\bibitem{cornell1} Хорстманн, Кей С., Корнелл, Гари. Java. Библиотека профессионала, Том 1. Основы. 9-е изд.
\bibitem{cornell2} Хорстманн, Кей С., Корнелл, Гари. Java. Библиотека профессионала, Том 1. Расширенные средства. 9-е изд.
\end{thebibliography}

\newpage
\appendix
\addtocontents{toc}{\vspace{5mm}\bfseries Приложения\par}
\footnotesize
\chapter{Тестирующий модуль}
Тестирующий модуль (код).
\chapter{Библиотека для разработки задач}
Библиотека для разработки задач (код).
\chapter{Модуль для работы с файловой системой}
Модуль для работы с файловой системой (код).
\chapter{Модуль с графическим интерфейсом}
Модуль с графическим интерфейсом (код).

\end{document}