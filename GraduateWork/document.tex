\begin{document}

\thispagestyle{empty}

{\large
\begin{center}
МИНОБРНАУКИ РОССИИ\par
ФЕДЕРАЛЬНОЕ ГОСУДАРСТВЕННОЕ БЮДЖЕТНОЕ\par
ОБРАЗОВАТЕЛЬНОЕ УЧРЕЖДЕНИЕ ВЫСШЕГО ПРОФЕССИОНАЛЬНОГО ОБРАЗОВАНИЯ\par
<<ВОРОНЕЖСКИЙ ГОСУДАРСТВЕННЫЙ УНИВЕРСИТЕТ>>\par
(ФГБОУ ВПО <<ВГУ>>)\par
\vspace{10mm}

Факультет прикладной математики, информатики и механики\par
\vspace{5mm}
Кафедра математического обеспечения ЭВМ\par
\vspace{22mm}

\textbf{Разработка задач для олимпиад по программированию}\par
\vspace{10mm}

Дипломная работа\par
\vspace{5mm}
по направлению 010500 <<Прикладная математика и информатика>>\par
\vspace{13mm}
\end{center}

Студент 4 курса 8 группы \underline{\qquad\qquad\qquad} Кузнецов~И.~Э.\par
\qquad\qquad\qquad\qquad\qquad\qquad\quad(подпись)\par
\vspace{2mm}
Зав. кафедрой \underline{\qquad\qquad\qquad} д.~ф.-м.~н.,~доц.~Махортов~С.~Д.\par
\qquad\qquad\qquad\qquad(подпись)\par
\vspace{2mm}
Руководитель \underline{\qquad\qquad\qquad} д.~ф.-м.~н.,~доц.~Махортов~С.~Д.\par
\qquad\qquad\qquad\qquad(подпись)\par
\vspace{23mm}

\begin{center}
Воронеж 2016
\end{center}}

\chapter*{Аннотация}
Текст аннотации.

%Алгоритмы неточного сопоставления строк среди всех строковых алгоритмов представляют собой целую группу важнейших и часто используемых нами в повседневности методов для решения возникающих перед нами задач. Наряду с точным сравненим строк их неточное сопоставление находит не менее важное применение во многих областях.

%Суть алгоритмов данной группы "--- поиск оптимального выравнивания строковых последовательностей, соответствующего тому или иному заданному критерию. В число задач, решаемых с помощью данных алгоритмов, входят поиск редакционного расстояния между строками, их сходства, их наибольшей общей подпоследовательности, локального выравнивания строк, а также поиск оптимального глобального выравнивания с учётом пропусков.

%Целью данной работы является реализация основных алгоритмов неточного сопоставления строк, решающих задачи, перечисленные выше. Предполагается создание приложения с графическим пользовательским интерфейсом на основе библиотеки Swing.

\renewcommand{\contentsname}{Содержание}
\tableofcontents

\chapter*{Введение}
\addcontentsline{toc}{chapter}{Введение}
Олимпиада по программированию "--- это интеллектуальное соревнование по решению различных задач на ЭВМ, для решения которых необходимо придумать и применить какой-либо алгоритм и программу на одном из языков программирования \cite{wiki}. Проводятся такие олимпиады сегодня как среди школьников, так и среди студентов и профессиональных работников, как на уровне вузов, так и на международном уровне. Многие крупные IT-компании, такие как Google, Facebook, Яндекс, Mail.ru, ВКонтакте, регулярно проводят свои собственные олимпиады, что, несомненно, говорит об их актуальности на сегодняшний день.

На олимпиаде по программированию участникам предлагается набор из нескольких задач, решением каждой задачи является программа, написанная на одном из разрешённых языков программирования. Программа должна считывать данные указанного формата из некоторого входного потока, обрабатывать их согласно условию задачи и выводить ответ в выходной поток. Чтобы решение было засчитано как верное, необходимо, чтобы оно выводило правильные ответы на заранее определённом наборе тестов.

Проверка решений на тестах производится с помощью специальных тестирующих систем. Такая система должна уметь компилировать код решений с помощью различных компиляторов, запускать решения на определённых наборах тестов, проверять корректность ответов простым сравнением с эталонными выходными данными или более сложными способами (например, с помощью специальных чекеров), а также "--- выносить вердикты по каждому полученному решению. Кроме того, система должна поддерживать различные системы оценивания решений (такие как IOI и ICPC \cite{wiki}). Наконец, от тестирующей системы требуется, чтобы она проверяла решения как можно быстрее, для чего, например, проверка может выполняться параллельно в нескольких потоках. Существует несколько известных тестирующих систем, используемых в реальных олимпиадах, например: Ejudge, PCMS2, Contester, Testsys.

Разработка задач для олимпиад по программированию "--- отдельный процесс, который также является весьма трудоёмким. В него входит написание корректного и содержательного условия и подготовка большого набора тестов, которые должны охватывать все возможные частные случаи в задаче. Тесты могут представлять собой данные очень больших объёмов, поэтому для их подготовки также необходимы специальное программное обеспечение и инструменты, такие как генераторы, валидаторы, чекеры и интеракторы, у каждого из которых есть своё особое предназначение. В качестве примера известной системы разработки задач можно привести систему Polygon \cite{polygon}, на которой подготавливают задачи для проведения контестов на платформе Codeforces \cite{codeforces}.

В данной работе мы поставим перед собой цель изучить основные аспекты работы тестирующих систем и систем разработки задач для олимпиад по программированию, а также "--- написать программу с удобным графическим пользовательским интерфейсом, предоставляющую возможность разрабатывать новые задачи и проверять решения на подготовленных тестах. Предполагается написание программы на языке Java с использованием библиотеки Swing.

\chapter{Общетеоретическая часть}
\section{Основные понятия}
На олимпиадном соревновании по программированию участникам предлагается решить несколько задач, каждая из которых представлена набором тестов "--- входных и выходных данных алгоритма, который нужно реализовать. Тесты подготавливаются заранее, и их подготовка "--- довольно длительный процесс. Он подразумевает генерацию входных данных для тестов, их валидацию, а также генерацию выходных данных, написание чекеров и иногда "--- интеракторов \cite{testlib}.

Полное решение задачи "--- работающая программа, которая на каждом тесте даёт правильный ответ. За время соревнования каждому участнику разрешается делать неограниченное количество посылок по каждой задаче различных версий своего решения в тестирующую систему, которая автоматически компилирует решения, запускает их на тестах и выносит для каждой посылки вердикт по определённым правилам. Таким образом, одна посылка участника "--- это файл с исходным кодом программы, вердикт, который тестирующая система присвоила этой посылке, а также "--- общая информация о посылке (максимальное время работы программе на одном тесте, количество использованной памяти и т. д.).

В зависимости от правил проведения того или иного соревнования, участникам могут быть видны вердикты по посылкам, которые они отправляют в тестирующую систему, либо не видны. Часто окончательным решением участника той или иной задачи считается та посылка, которую он отослал позже остальных. Кроме того, участникам может быть доступен общий монитор соревнования, в котором видны текущие результаты по каждой задаче всех остальных участников.

По определённым правилам, в зависимости от используемой на соревновании системы оценивания (ICPC, IOI \cite{wiki}), каждой задаче присваивается некоторое количество баллов. Они суммируются для каждого участника, и на основе них затем формируется итоговая таблица результатов соревнования.

Таким образом, разработка задач и проверка решений "--- два взаимосвязанных процесса, но рассматривать их можно отдельно друг от друга. Исследуем каждый из них более подробно.
\section{Правила проведения олимпиад}
\input{rules}
\section{Инструменты разработки задач}
\input{development_tools}
\section{Запуск решений}
\input{decision_running}

\chapter{Практическая часть}
\section{Постановка задачи}
\input{task_statement}
\section{Средства реализации}
\input{realization_tools}
\section{Тестирующий модуль}
\input{testing_module}
\section{Библиотека для разработки задач}
\input{development_module}
\section{Модуль для работы с файловой системой}
Модуль для работы с файловой системой.
\section{Модуль с графическим интерфейсом}
\input{interaction_module}
\section{Пример работы приложения}
\input{example}

\chapter*{Заключение}
\addcontentsline{toc}{chapter}{Заключение}
Текст заключения.

%В ходе курсовой работы было исследовано четыре различных алгоритма поиска выравниваний пар строк, обладающих определёнными характеристиками. Мы подробно обсудили, как используется в них принцип динамического программирования и каким образом строится искомое оптимальное выравнивание, поговорили об эффективности каждого из этих алгоритмов, а также дополнительно для двух из них выявили способ нахождения количества кооптимальных выравниваний.

%Была поставлена задача написать приложение, реализующее все четыре алгоритма и позволяющее запускать их с некоторым набором входных параметров. В ходе работы это приложение было разработано, в процессе чего активно использовались объектно-ориентированное программирование и паттерны проектирования. Была подробно описана структура классов, использованных в приложении, и приведён пример его работы.

%Ясно, что в созданном приложении реализованы лишь некоторые алгоритмы, работающие с выравниваниями строковых последовательностей, и что помимо них существует ещё множество подобных алгоритмов, из чего следует возможность дальнейшей разработки данного приложения.

\renewcommand{\bibname}{Список использованных источников}
\begin{thebibliography}{9}
\addcontentsline{toc}{chapter}{Список использованных источников}
%\bibitem{gasfild} Гасфилд Дэн. Строки, деревья и последовательности в алгоритмах: Информатика и вычислительная биология / Пер. с англ. И.~В.~Романовского.~"--- СПб.~: Невский Диалект; БХВ-Петербург, 2003.~"--- 654~с.
%\bibitem{smith} Смит, Билл. Методы и алгоритмы вычислений на строках.~: Пер. с англ.~"--- М.~: ООО <<И.~Д.~Вильямс>>, 2006.~"--- 496~с.
%old one
%\bibitem{wiki_dist} Wikipedia, статьи.\par
%(URL: https://ru.wikipedia.org/wiki/Выравнивание\_последовательностей)\par
%(URL: https://ru.wikipedia.org/wiki/Расстояние\_Левенштейна)
\bibitem{gamma} Приёмы объектно-ориентированного проектирования. Паттерны проектирования. Гамма Э. [и др.].~"--- СПб~: Питер, 2001.~"--- 368~с.
\bibitem{java} Java, документация.\par
(URL: http://docs.oracle.com/javase/8/docs/)
\bibitem{cornell1} Хорстманн, Кей С., Корнелл, Гари. Java. Библиотека профессионала, Том 1. Основы. 9-е изд.
\bibitem{cornell2} Хорстманн, Кей С., Корнелл, Гари. Java. Библиотека профессионала, Том 1. Расширенные средства. 9-е изд.
\end{thebibliography}

\newpage
\appendix
\addtocontents{toc}{\vspace{5mm}\bfseries Приложения\par}
\footnotesize
\chapter{Тестирующий модуль}
\input{testing_module_code}
\chapter{Библиотека для разработки задач}
Библиотека для разработки задач (код).
\chapter{Модуль для работы с файловой системой}
Модуль для работы с файловой системой (код).
\chapter{Модуль с графическим интерфейсом}
Модуль с графическим интерфейсом (код).

\end{document}