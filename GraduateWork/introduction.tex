Текст введения.

%Алгоритмы неточного сопоставления строк представляют собой целую группу важнейших и часто используемых нами в повседневности методов для решения возникающих перед нами задач. Кратко перечислим известные применения алгоритмов неточного сопоставления, чтобы подчеркнуть их нынешнюю актуальность.

%Наиболее часто упоминаемое применение данных алгоритмов "--- в области биоинформатики. Дэн Гасфилд в своей книге \cite{gasfild} подчёркивает важность некоего <<первого факта анализа биологических последовательностей>>, позволяющего применять строковые алгоритмы к биологическим последовательностям и получать в результате выводы, имеющие большую научную ценность. Вот как Дэн Гасфилд формулирует данный факт: <<В биомолекулярных последовательностях (ДНК, РНК или аминокислотных последовательностях) высокое сходство последовательностей обычно влечёт существенное функциональное или структурное сходство>>.

%Биомолекулярные последовательности можно представлять в виде длинных цепочек из азотистых оснований (которых может быть всего четыре -- аденин, гуанин, цитозин и тимин). Таким образом, к ним можно применять все строковые алгоритмы. Однако заметим, что применение к этим последовательностям алгоритмов неточного сопоставления позволяет решить некоторые проблемы, которые точное сравнение решить бы не смогло, а именно "--- <<наличие ошибок в молекулярных данных>> и <<активные мутационные процессы>> ДНК \cite{gasfild}.

%Сегодня существуют целые базы данных ДНК, содержащие в себе огромные объёмы информации о биомолекулярных последовательностях. После расшифровки нового гена принято переводить его в последовательность аминокислот и проверять на сходство с записями баз данных. Как пишет Дэн Гасфилд \cite{gasfild}, <<Сейчас никому в голову не придёт публиковать расшифровку вновь клонированного гена без проведения такого поиска>>. Таким образом, неточное сопоставление строковых последовательностей занимает в биоинформатике центральное место.

%Но, конечно, кроме этого насчитывается также ещё множество различных применений данной группы алгоритмов. Они активно применяются для исправления ошибок в словах в поисковых системах, базах данных, при вводе текста и при автоматическом распознавании отсканированного текста или речи. Подобное исправление ошибок стало привычным в современном мире, где почти у каждого человека есть свой смартфон, которым пользуются изо дня в день. Кроме того, алгоритмы неточного сравнения строк используются также для сравнения текстовых файлов, где в качестве символов выступают строки, а в качестве строк "--- файлы \cite{wiki_dist}.

%Мы будем рассматривать данные алгоритмы без жёсткой привязки к какому-либо их конкретному применению. Наше исследование будет носить сугубо теоретический характер и будет относиться к сравнению строк вообще, каких бы то ни было. Нами будет рассмотрено и реализовано несколько известных алгоритмов неточного сравнения пар строк, будет приведено их краткое теоретическое описание и оценена их временная сложность.